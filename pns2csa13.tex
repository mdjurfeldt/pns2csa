\documentclass[a4paper]{article}

\usepackage{authblk}
\renewcommand\Affilfont{\scriptsize}

\usepackage{natbib}

\title{Modeling connectivity: Connection-set Algebra in NEST and PyNN}
\author[1,2]{Mikael Djurfeldt}
\author[3]{Andrew P. Davison}
\author[4]{Jochen M. Eppler}
\affil[1]{PDC, KTH, Stockholm, Sweden}
\affil[2]{International Neuroinformatics Coordinating Facility (INCF),
  Stockholm, Sweden}
\affil[3]{Unit{\'e} de Neurosciences, Information et Complexit{\'e}
  (UNIC), CNRS, Gif sur Yvette, France}
\affil[4]{Institute of Neuroscience and Medicine (INM­6) and
  Institute for Advanced Simulation (IAS­6),
  J{\"u}lich Research Centre and JARA, J{\"u}lich, Germany}

\begin{document}

\maketitle

\begin{abstract}
  Most neuronal network simulators provide simulator­specific methods
  for specifying connectivity.  In addition, several universal
  description languages for neuronal network models (e.g. PyNN,
  NeuroML and NineML) have been developed in the recent
  years. Usually, there is a choice between explicitly specifying
  individual connections or selecting one of a predefined set of
  connection primitives, such as different forms of random
  connectivity. Common primitives are well suited for building random
  balanced networks, synfire chains and recurrent networks without
  much structure. However, they often either lack the required
  expressiveness or their implementations lack the required
  computational efficiency to specify the connectivity of complex
  hierarchical network models now under investigation.

  The connection­set algebra \citep[CSA;][]{djurfeldt12} is a general
  formalism for specifying the connectivity of neuronal network
  models. CSA provides operators to form more complex sets of
  connections from simpler ones and also provides parameterization of
  such sets, and thus allows to express connectivity from its
  small­scale to its large­scale structure. The CSA is expressive
  enough to describe a wide range of connectivities and can serve as a
  concise notation for network structure in scientific writing as well
  as in model description code for neural simulations. CSA
  implementations allow for scalable and efficient representation of
  connectivity in parallel neuronal network simulators. The
  expressiveness of CSA makes prototyping of network structure easy.

  Here, we describe the Python implementation of CSA and its use in
  the context of the neural simulation tool NEST
  \citep{Gewaltig_07_11204} and the simulator­independent
  specification language PyNN \citep{Davison09}. We developed an
  interface (the ConnectionGenerator) that abstracts the simulator and
  the connection generating library from each other, making both the
  simulator and the connectivity description library replaceable. We
  demonstrate how CSA can be used for the specification of
  connectivity in PyNN and how using CSA as a high­level
  representation of connectivity patterns when transferring the model
  description to NEST avoids performance penalties and results in good
  scalability.
\end{abstract}

\section{Introduction}

%CSA

\subsection{NEST}


%%Topology

%PyNN

%NineML

%NeuroML

\section{The Connection-set Algebra}

%basic concepts

%examples

%parallelization/implementation/Python wrapper

\section{The ConnectionGenerator API}

%idea: free to choose simulator / connection generating library

%JE makes figure

\section{NEST connection generators}

%JE writes this

\subsection{Index translation}

\section{PyNN connection generators}

%AD? JE?

\section{Benchmarks}

%test scaling

\section{Discussion}

\bibliographystyle{neuralcomput_natbib}
\bibliography{pns2csa13}

\end{document}
