
\documentclass{frontiersSCNS} % for Science articles

\usepackage{url,lineno}
\linenumbers

\usepackage{ucs}
\usepackage[utf8x]{inputenc}

\copyrightyear{}
\pubyear{}

\def\journal{Neuroinformatics}%%% write here for which journal %%%
\def\DOI{}
\def\articleType{Research Article}
\def\keyFont{\fontsize{8}{11}\helveticabold }
\def\firstAuthorLast{Djurfeldt {et~al.}}
\def\Authors{Mikael Djurfeldt\,$^{1,2,*}$, Andrew P. Davison\,$^{3}$
  and Jochen M. Eppler\,$^4$}
% Affiliations should be keyed to the author's name with superscript numbers and be listed as follows: Laboratory, Institute, Department, Organization, City, State abbreviation (USA, Canada, Australia), and Country (without detailed address information such as city zip codes or street names).
% If one of the authors has a change of address, list the new address below the correspondence details using a superscript symbol and use the same symbol to indicate the author in the author list.
\def\Address{$^{1}$PDC, KTH, Stockholm, Sweden\\
  $^{2}$International Neuroinformatics Coordinating Facility (INCF),
  Stockholm, Sweden\\
  $^{3}$Unité de Neurosciences, Information et Complexité (UNIC),
  CNRS, Gif sur Yvette, France\\
  $^{4}$Institute of Neuroscience and Medicine (INM-6) and Institute
  for Advanced Simulation (IAS-6), Jülich Research Centre and
  JARA, Jülich, Germany}
% The Corresponding Author should be marked with an asterisk
% Provide the exact contact address (this time including street name and city zip code) and email of the corresponding author
\def\corrAuthor{Mikael Djurfeldt} \def\corrAddress{International
  Neuroinformatics Coordinating Facility (INCF), Nobels väg 15 A,
  Stockholm, SE-17177, Sweden} \def\corrEmail{djurfeldt@incf.org}

% \color{FrontiersColor} Is the color used in the Journal name, in the title, and the names of the sections


\begin{document}
\onecolumn
\firstpage{1}

\title[CSA in NEST and PyNN]{Modeling connectivity: Connection-set Algebra in NEST and PyNN}
\author[\firstAuthorLast ]{\Authors}
\address{}
\correspondance{}
\extraAuth{}% If there are more than 1 additional author, comment this line and uncomment the next one
%\extraAuth{corresponding Author2 \\ Laboratory X2, Institute X2, Department X2, Organization X2, Street X2, City X2 , State XX2 (only USA, Canada and Australia), Zip Code2, X2 Country X2, email2@uni2.edu}
\topic{Python in Neuroscience 2}

\maketitle
\begin{abstract}

Most neuronal network simulators provide simulator­specific methods
for specifying connectivity.  In addition, several universal
description languages for neuronal network models (e.g. PyNN,
NeuroML and NineML) have been developed in the recent
years. Usually, there is a choice between explicitly specifying
individual connections or selecting one of a predefined set of
connection primitives, such as different forms of random
connectivity. Common primitives are well suited for building random
balanced networks, synfire chains and recurrent networks without
much structure. However, they often either lack the required
expressiveness or their implementations lack the required
computational efficiency to specify the connectivity of complex
hierarchical network models now under investigation.

The connection­set algebra \citep[CSA;][]{djurfeldt12} is a general
formalism for specifying the connectivity of neuronal network
models. CSA provides operators to form more complex sets of
connections from simpler ones and also provides parameterization of
such sets, and thus allows to express connectivity from its
small­scale to its large­scale structure. The CSA is expressive
enough to describe a wide range of connectivities and can serve as a
concise notation for network structure in scientific writing as well
as in model description code for neural simulations. CSA
implementations allow for scalable and efficient representation of
connectivity in parallel neuronal network simulators. The
expressiveness of CSA makes prototyping of network structure easy.

Here, we describe the Python implementation of CSA and its use in
the context of the neural simulation tool NEST
\citep{Gewaltig_07_11204} and the simulator­independent
specification language PyNN \citep{Davison09}. We developed an
interface (the ConnectionGenerator) that abstracts the simulator and
the connection generating library from each other, making both the
simulator and the connectivity description library replaceable. We
demonstrate how CSA can be used for the specification of
connectivity in PyNN and how using CSA as a high­level
representation of connectivity patterns when transferring the model
description to NEST avoids performance penalties and results in good
scalability.

\tiny
 \keyFont{ \section{Keywords:} model description, connectivity,
 neural simulation, Python, large-scale modeling } %All article
 %types: you may provide up to 8 keywords; at least 5 are mandatory.
\end{abstract}

\section{Introduction}

%CSA

\subsection{NEST}


%%Topology

%PyNN

%NineML

%NeuroML

\section{The Connection-set Algebra}

%basic concepts

%examples

%parallelization/implementation/Python wrapper

\section{The ConnectionGenerator API}

%idea: free to choose simulator / connection generating library

%JE makes figure

\section{NEST connection generators}

%JE writes this

\subsection{Index translation}

\section{PyNN connection generators}

%AD? JE?

\section{Benchmarks}

%test scaling

\section{Discussion}


% \textbf{Figure 1.}{ Enter the caption for your figure here.  Repeat as  necessary for each of your figures.}\label{fig:01}% Don't add the figures in the LaTeX files, please upload them when submitting the article. Frontiers will add the figures at the end of the provisional pdf.
% 
% \begin{table}[!t]
% \processtable{Resolution Requirements for the figures\label{Tab:01}}
% {\begin{tabular}{lllll}\toprule
% Image Type & Description & Format & Color Mode & Resolution\\\midrule
% Line Art & An image composed of lines and text,  & TIFF, EPS, JPEG & RGB, Bitmap & 900 - 1200 dpi\\
%            & which does not contain tonal or shaded areas.& & &\\
%            Halftone & A continuous tone photograph, which contains no text. & TIFF, EPS, JPEG & RGB, Grayscale & 300 dpi\\
% Combination & Image contains halftone + text or line art elements. & TIFF, EPS, JPEG & RGB,Grayscale & 600 - 900 dpi\\\botrule
% \end{tabular}}{This is a footnote}
% \end{table}

\section*{Conflict-of-Interest Statement}
The authors declare that the research was conducted in the absence of
any commercial or financial relationships that could be construed as a
potential conflict of interest.

\section*{Acknowledgement}
Partially supported by the Helmholtz Association: HASB and portfolio
theme SMHB, the Jülich Aachen Research Alliance (JARA), the VSR
computation time grant JINB33 on the JUQUEEN supercomputer in Jülich,
the Next-Generation Supercomputer Project of MEXT, and EU Grant 269921
(BrainScaleS).

%\paragraph{Funding\textcolon} Text Text Text Text Text Text  Text Text.

%\section*{Supplemental Data}

\bibliographystyle{frontiersinSCNS&ENG} % for Science and Engineering articles
\bibliography{pns2csa13}

\end{document}
